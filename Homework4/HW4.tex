\section*{1}
\begin{enumerate}[label=\alph*.]
    \item eBGP.
    \item iBGP.
    \item eBGP.
    \item iBGP.
    \item Since OSPF is being used, l1 will be stored since it has a lower cost (9 vs 10) than l2.
    \item Since the new cost via l2 is 5, which is lower than the cost via l1, l2 will replace l1.
    \item l1 will be used since it has a shorter AS-PATH length.
\end{enumerate}

\section*{2}
\begin{enumerate}[label=\alph*.]
    \item Since E and F are in the same local network, routing through R1 is not required. Instead the packets will reach F through S3.
    
    Source IP: E's IP address
    
    Destination IP: F's IP address
    
    Source MAC: E's MAC address
    
    Destination MAC: F's MAC address
    
    \item Since B is not in E's local network, it will not do an ARP query. 
    
    Source IP: E's IP address
    
    Destination IP: B's IP address
    
    Source MAC: E's MAC address
    
    Destination MAC: MAC address of R1 facing S3
    
    \item Once S1 recieves the ARP request message, it will add A's IP and MAC address to its forwarding table and forwards the packet to S2.
    
    R1 will recieve the message, however it will not forward the message.
    
    B does not need to ask for A's MAC address since the information is included in the ARP query message.
    
    Since S0 already knows A's MAC address (from forwarding the query packet), S0 can forward the packet from B directly to A without involving S1.
\end{enumerate}

\section*{3}
Dividend = D || x = \texttt{10011001000}\\
Divisor = G = \texttt{1001}\\

After dividing, we get\\
Result = \texttt{10001000}\\
Remainder = \texttt{0}\\
CRC bits = \texttt{000}\\