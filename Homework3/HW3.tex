\section*{1}

Data in each fragment = $500 - 20$ = \SI{480}{bytes}.

Number of fragments = $\frac{1600}{480} = 3.\overline{3}$. Therefore, we will require 4 fragments to transfer the data.

All the fragments will have the identification number set to 291.
All the fragments other than the last fragment will have length of 500 bytes and frag flag set to 1. The last fragment will have a length of 160 bytes and frag flag set to 0. The offset values of the 4 packets are 0, 60, 120, and 180 respectively.

\section*{2}
\begin{enumerate}[label=\alph*.]
    \item We can count the number of unique hosts using the identification numbers. Since the ID numbers are generated sequentially from a random number, the chances of a collision of the initial number are very low. Hence, we can count the number of "strands" of ID numbers to find out the number of hosts.
    
    e.g. If host 1 starts from ID 10 and host 2 starts for ID 500, packets from Host 1 will have IDs 10, 11, 12, ... and Host 2 will have IDs 500, 501, 502, .... So we have two discrete strands of IDs here meaning there are two unique hosts.
    
    \item If the ID numbers are random then the above method will not work as it will not be possible to form "strands" of ID numbers.
\end{enumerate}

\section*{3}
The router will use DHCP to assign internal IP address to all the 5 computers. Every computer will go through the \texttt{DISCOVER}, \texttt{OFFER}, \texttt{REQUEST}, \texttt{ACK} procedure with the router and establish an IP for the network.

Yes, the router will have to use a NAT since we only receive one IP address from the ISP and we have multiple devices connecting to the router simultaneously.

\section*{4}

\begin{center}
    \begin{tabular}{|c|c|c|c|c|c|c|}
        \hline
        \textbf{Step} & \textbf{N'} & \textbf{D(u),p(u)} & \textbf{D(v),p(v)} & \textbf{D(w),p(w)} & \textbf{D(y),p(y)} & \textbf{D(z),p(z)} \\ 
        \hline
        0 & x & 1,x & 6,x & 5,x & 4,x & $\infty$ \\  
        \hline
        1 & xu & 1,x & 5,u & 5,x & 4,x & 2,u \\  
        \hline
        2 & xuz & 1,x & 5,u & 5,x & 3,z & 2,u \\  
        \hline
        3 & xuzy & 1,x & 5,u & 4,y & 3,z & 2,u \\  
        \hline
        4 & xuzyw & 1,x & 5,u & 4,y & 3,z & 2,u \\  
        \hline
        5 & xuzywv & 1,x & 5,u & 4,y & 3,z & 2,u \\  
        \hline
    \end{tabular}
\end{center}

\section*{5}
\begin{enumerate}[label=\alph*.]
    \item ~
    \begin{table}[h]
        \centering
        \begin{tabular}{|l|l|}
            \hline
                \multirow{2}{*}{Router z} & Informs w, Dz(x)=$\infty$   \\ \cline{2-2} 
                                          & Informs y, Dz(x)=6          \\ \hline
                \multirow{2}{*}{Router w} & Informs y, Dw(x)=$\infty$   \\ \cline{2-2} 
                                          & Informs z, Dw(x)=5          \\ \hline
                \multirow{2}{*}{Router y} & Informs w, Dy(x)=4          \\ \cline{2-2} 
                                          & Informs z, Dy(x)=4          \\ \hline
        \end{tabular}
    \end{table}

    \item Yes there will be a count-to-infinity problem.
    
     \begin{table}[h]
        \hskip-2.0cm
        \begin{tabular}{|c|c|c|c|c|c|}
        \hline
        \textbf{time} & \textbf{t0} & \textbf{Round 1} & \textbf{Round 2} & \textbf{Round 3} & \textbf{Round 4} \\
        \hline
        Z & & & No Change & \makecell{Update w Dz(x)=$\infty$\\Update y Dz(x)=$11$} & \\
        \hline
        W & & & \makecell{Update y, Dw(x)=$\infty$\\Update z, Dw(x)=$10$} & & No Change\\
        \hline
        Y & Dy(x)=9 & \makecell{Update w, Dy(x)=9\\Update z, Dy(x)=$\infty$} & & No Change & \makecell{Update w, Dy(x)=14\\Update z, Dy(x)=$\infty$}\\
        \hline
        \end{tabular}
    \end{table}
    
    We can see that Round 1 and 4 are similar where \texttt{Y}'s distance to \texttt{X} increases. This would increase every 4th round leading to a count-to-infinity problem.
    
\end{enumerate}