\section*{1}
\begin{enumerate}
  \item $1*RTT =\SI{100}{ms}$
  \item $Throughput = \frac{max-data}{RTT} = \frac{1000 * 6 * 8}{0.1} = \SI{480}{Kbps}$
  \item $size = throughput * RTT = \frac{\num{e8}}{8} = \SI{1.25e7}{bytes}$
\end{enumerate}

\section*{2}
\begin{equation*}
\begin{split}
  01101001 11110110\\
 +11100011 00011100\\
 =0100110100000011 \text{(1 wrap around)}\\
 +10101010 10101010\\
 =1111011110111101
\end{split}
\end{equation*}
Taking 1's complement, the internet checksum will be $0000100001000010$.

\section*{3}
\begin{enumerate}[label=\alph*.]
    \item 1RTT to increase to 7MSS, 2RTT to increase to 8MSS. Following this pattern, we can extrapolate to see we will need 6RTT to increase to 12MSS.
    \item Total MSS sent = $6 + 7 + 8 + 9 + 10 + 11 + 12 = \SI{51}{MSS}$
    
    Average throughput = $\frac{\SI{51}{MSS}}{\SI{6}{RTT}} = \SI{8.5}{MSS/RTT}$
\end{enumerate}

\section*{4}
\begin{enumerate}
    \item Number of packets sent during this period,
    \begin{equation*}
        \begin{split}
            \frac{W}{2}+(\frac{W}{2} + 1)+...+W=\sum_{n=0}^{W/2}(\frac{W}{2} + n)\\
            = (\frac{W}{2} + 1)\frac{W}{2} + \frac{\frac{W}{2}*\frac{W}{2} + 1}{2}\\
            = \frac{3}{8}W^2 + \frac{3}{4}W
        \end{split}
    \end{equation*}
    $$\therefore \text{Loss rate} = \frac{1}{\frac{3}{8}W^2 + \frac{3}{4}W}$$
    
    \item On average, $W$ is very large, hence $\frac{3}{8}W^2 \gg \frac{3}{4}W$. Therefore, $L \thickapprox \frac{8}{3W^2} \Rightarrow W \thickapprox \sqrt{\frac{8}{3L}}$.
    
    Average throughput $= 0.75*\sqrt{\frac{8}{3L}} = \frac{1.22*MSS}{RTT*\sqrt{L}}$
    
    \item \begin{equation*}
            \begin{split}
                \SI{1}{Gbps} = 0.75*\sqrt{\frac{8}{3L}} = \frac{1.22*1500*8}{\sqrt{L}*0.1}\\
                \Rightarrow \sqrt{L} = \num{14640e-9}\\
                \therefore L = \num{2.1433e-10}
            \end{split}
        \end{equation*}
        
        For 100Gbps, 
        \begin{equation*}
            \begin{split}
                \SI{100}{Gbps} = 0.75*\sqrt{\frac{8}{3L}} = \frac{1.22*1500*8}{\sqrt{L}*0.1}\\
                \Rightarrow \sqrt{L} = \num{14640e-11}\\
                \therefore L = \num{2.1433e-14}
            \end{split}
        \end{equation*}
\end{enumerate}

\section*{5}
\begin{enumerate}
    \item The two connections are in the "congestion avoidance" state.
    
    \texttt{cwnd} size of connection 1 after loss event = $\frac{80}{2} + 3 = \SI{43}{KB}$.
    
    \texttt{cwnd} size of connection 2 after loss event = $\frac{40}{2} + 3 = \SI{23}{KB}$.
    
    \item Since the connection is in "congestion avoidance" state, cwnd should increase by 1 MSS every 1 RTT. We can see that the connection increases by 10 MSS (10 KB) every 1 second.
    
    Therefore RTT = $\frac{1}{10} = \SI{0.1}{seconds}$.
    
    Average for connection 1 = $\frac{(50 + 80) * 8 * 10^3}{2*0.1} = \SI{5.2}{Mbps}$
    
    Average for connection 2 = $\frac{(10 + 40) * 8 * 10^3}{2*0.1} = \SI{2}{Mbps}$
    
    \item For both connections, $$\frac{3}{4} * \frac{60 * 8 * 10^3}{0.1} = \SI{3.6}{Mbps}$$
    
    \item For connection 1, \texttt{cwnd} will move between 20 KB and 40 KB.
    $$\therefore \frac{3}{4} * \frac{40 * 8 * 10^3}{0.2} = \SI{1.2}{Mbps}$$
    
    For connection 2, \texttt{cwnd} will move between 40 KB and 80 KB.
    $$\therefore \frac{3}{4} * \frac{80 * 8 * 10^3}{0.1} = \SI{4.8}{Mbps}$$
\end{enumerate}